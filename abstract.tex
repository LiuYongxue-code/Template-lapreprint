\abstract{Interest in the pathophysiology, etiology, management, and outcomes of patients with tricuspid regurgitation (TR) has
grown in the wake of multiple natural history studies showing progressively worse outcomes associated with increasing
TR severity, even after adjusting for multiple comorbidities. Historically, isolated tricuspid valve surgery has been
associated with high in-hospital mortality rates, leading to the development of transcatheter treatment options. The aim
of this first Tricuspid Valve Academic Research Consortium document is to standardize definitions of disease etiology and
severity, as well as endpoints for trials that aim to address the gaps in our knowledge related to identification and
management of patients with TR. Published by Elsevier Inc on behalf of the American College of Cardiology and The Society of
Thoracic Surgeons and by Oxford University Press on behalf of the European Society of Cardiology. Standardizing endpoints for trials should provide consistency and enable mean-
ingful comparisons between clinical trials. This is an open
access article under the CC BY-NC-ND license (\href{http://creativecommons.org/licenses/by-nc-nd/4.0/}{(http://creativecommons.org/licenses/by-nc-nd/4.0/})
}